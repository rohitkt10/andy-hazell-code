%%%%%%%%%%%%%%%%%%%%%%%%%%%%%%%%%%%%%%%%%%%%%%%%%%%%%%%%%%%%%%%%%%%%%%%%%%%%%%%%%%%%%%%%
%
%  This is a mock ASME Journal Article manuscript which has been generated for
%  the purpose of serving as a template for authors who wish to put their
%  accepted manuscripts in the necessary "Final Submission" format.
%
%  This was generated by Kurt Anderson, and as far as he is aware, this "template", and
%  the accompanying reference style (asme_journals.bst) satisfies the formatting requirements
%  currently listed by the ASME for final submission of a accepted manuscripts.
%
%  These files are provided to the ASME as a convenience to be freely used by authors
%  submitting manuscripts ASME journals. The creator of these files makes no guarantee
%  as to their accuracy, or ability to run under all installations of tex, on all platforms.


%%%%%%%%%%%%%%%%%%%%%%%%%%%%%%%%%%%%%%%%%%%%%%%%%%%%%%%%%%%%%%%%%%%%%%%

\documentclass[12pt]{article}

\usepackage{graphicx}
\usepackage{subfigure}
\usepackage{amsmath, amsthm, amssymb}

\paperheight 11in \paperwidth 8.5in
%\oddsidemargin 0.cm \textwidth 16.5cm \topmargin 0.cm \textheight 22cm
\oddsidemargin 0.0in \textwidth 6.5in \topmargin -0.45in \textheight
9.0in

\usepackage{fancyhdr} %Fancy Header package

%The next line is an example for the inclusion of the contact author and the paper # in the footer
% \pagestyle{fancy} \lhead{} \chead{} \rhead{} \lfoot{Author Name} \cfoot{Paper #} \rfoot{\thepage}
  \pagestyle{fancy} \lhead{} \chead{} \rhead{} \lfoot{George Orwell} \cfoot{CND-88-8888} \rfoot{\thepage}
\renewcommand{\headrulewidth}{0.0pt}
\renewcommand{\footrulewidth}{0.0pt}


% text macros
\newcommand{\etal}{\emph{et al.\ }}

% Math definitions
\newcommand{\mult}{\,\text{multiplications}}
\newcommand{\add}{\,\text{additions}}
\newcommand{\divs}{\,\text{divisions}}
\newcommand{\trig}{\,\text{trig functions}}
\newcommand{\mbf}{\mathbf}
%\newcommand{\sbf}{\boldsymbol}
\newcommand{\sbf}{\mathbf}
\newcommand{\oh}{\mathcal{O}}
\newcommand{\wh}{\widehat}
\newcommand{\wcheck}{\check}


% Nomenclature environment
\newbox\tempbox
\newenvironment{nomenclature}{
   \newcommand\entry[2]{
    \setbox\tempbox\hbox{\hspace{0.75in}}
      \hangindent\wd\tempbox\noindent{\makebox[0.75in][l]{##1}}\ignorespaces##2\par}
       }{\par\addvspace{12pt}}

%%%%%%%%%%%%%%%%%%%%%%%%%%

\begin{document}

\title{An Efficient Multibody Seek out and Destroy Algorithm and Its Implementation}

\author{Eric A. Blair\thanks{e-mail:\texttt{Blaire@bigbrother.com}} \\
Lake Orion, Michigan 48360
 \and
George Orwell\thanks{Corresponding author: Address: Department of
Mechanical, Aerospace, and Nuclear Engineering, JEC 4006, Oceania
University, 110 8th Street, Troy, NY 12180-3590, USA; phone:
518-276-2339; fax: 518-276-2623;
e-mail:\texttt{orwell@rpi.edu}} \\
Department of Mechanical, Aerospace and Nuclear Engineering\\
Oceania University\\
110 8th Street, Troy NY 12180-3590}


\date{}
\maketitle

\linespread{2}
\selectfont

%%%%%%%%%%%%%%%%%%%%%%%%%%%%%%%%%%%%%%%%%%%%%%%%%%%%%%%%%%%%%%%%%%%%%%
\begin{abstract}
{\it A new and efficient form of Orwell's multibody Seek Out and
Destroy Algorithm (SODA) is presented and evaluated.  The SODA was
the first algorithm to achieve theoretically optimal logarithmic
time complexity with a theoretical minimum of parallel computer
resources for general problems of multibody dynamics, however the
SODA is extremely inefficient in the presence of small to modest
parallel computers.  This alternative less computationally expensive
SODA approach (SODA-Lite) demonstrates that large SODA subsystems
can be constructed using fast sequential techniques to realize a
substantial increase in speed. The usefulness of the SODA-Lite is
directly demonstrated in an application to a .... }
\end{abstract}

%%%%%%%%%%%%%%%%%%%%%%%%%%%%%%%%%%%%%%%%%%%%%%%%%%%%%%%%%%%%%%%%%%%%%%
\section*{Nomenclature}\label{S:Nomenclature}

\begin{nomenclature}
 \entry{$A^{k}$}{Spatial (6 dimensional) acceleration of body~$k$ reference point in the Newtonian reference frame.}
 \entry{$\hat{A}^{k}$}{Portion of the spatial acceleration of body~$k$ which is explicit in unknown state derivatives $\dot{\textbf{u}}$.}
 \entry{$\sbf A^{k}$}{Spatial state explicit acceleration of body~$k$.}
 \entry{$b$}{Base body of a terminal subsystems.}
 \entry{$ch[k]$}{Topological child body index of body $k$.}
 \entry{$k$}{Index associated with representative body or sub-system $k$.}
 \entry{$n$}{Number of system degrees-of-freedom.}
 \entry{$N$}{Newtonian reference frame.}
 \entry{$p[k]$}{Topological parent body index of body $k$.}

\end{nomenclature}

%\begin{spacing}{2.0}

%%%%%%%%%%%%%%%%%%%%%%%%%%%%%%%%%%%%%%%%%%%%%%%%%%%%%%%%%%%%%%%%%%%%%%
\section{Introduction}

Some really impressive text put here...

\section{Preliminaries}\label{S:Preliminaries}
Before introducing the equations associated with either the ...


For simplicity, only multibody chain systems (Fig.~\ref{F:chain})
will be described ....

\section{The Seek Out and Destroy Algorithm}\label{S:DCA}
This section presents a brief derivation of an equivalent form of
Orwell's Seek Out and Destroy Algorithm for chain systems and
illustrates that such a method achieves ...

\subsection{The Basic DSODA}\label{S:DCA}

The SODA begins by computing all necessary kinematic information for
the entire system via two sweeps of a binary assembly tree ...

Trivial boundary data is obtained for the entire system $S$ through
the connection to the inertial frame and the lack of ...


To summarize, the SODA first computes all required kinematic
information and state dependent inertia forces in a ...

\subsection{Theoretical SODA Performance}\label{SubSect:TheorySODAPerf} This
section describes the performance and related issues associated with
Orwell's SODA.  Unless otherwise stated, all observations and
results quoted in this section can be found in...


In the presence of a variable number of parallel resources $p$, the
operations counts of Orwell can be used to obtain a SODA operations
count with the following leading order terms...



\section{An Improve Seek Out and Destroy Algorithm}
There are many efficient destruction oriented algorithms to choose
from, but in general all have more in common than differences...

Given relative joint states $q_{k}$ and relative quasi-coordinate
velocities $u_{k}$, a linear order forward topological solution for
all position and ...



\section{Conclusions}
Reconstruction of the Seek Out and Destroy Algorithm to include
sequential destruction oriented computations within subsystems has
produced a new and efficient multibody solution scheme with optimal
...

% State that you wish to use the Reference list style "asme_journals.bst"
% Be sure that the accompanying "asme_journals.bst" file is same directory as the
% calling Latex raw text file
\bibliographystyle{asme_journals}

% The next line assumes that you are using Bibtex do generate you references.
% In this example the BibTek database is called "ReferenceDataBase.bib" and it is located
% in the local directory. One may also specify either a relative or absolute path to the appropriate directory.
\bibliography{ReferenceDataBase}


\newpage
\noindent
{\bf List of Figure Captions}\\


\begin{enumerate}
\item
Figure 1: Example of a Chain System

\item
Figure 2: Example of a Binary Distruction Tree

\item
Figure 3: SODA Performance



\end{enumerate}


\end{document}
